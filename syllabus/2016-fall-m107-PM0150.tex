\documentclass[11pt,]{article}
\usepackage[margin=1in]{geometry}
\newcommand*{\authorfont}{\fontfamily{phv}\selectfont}
\usepackage[]{mathpazo}
\usepackage{abstract}
\renewcommand{\abstractname}{}    % clear the title
\renewcommand{\absnamepos}{empty} % originally center
\newcommand{\blankline}{\quad\pagebreak[2]}

\providecommand{\tightlist}{%
  \setlength{\itemsep}{0pt}\setlength{\parskip}{0pt}} 
\usepackage{longtable,booktabs}

\usepackage{parskip}
\usepackage{titlesec}
\titlespacing\section{0pt}{12pt plus 4pt minus 2pt}{6pt plus 2pt minus 2pt}
\titlespacing\subsection{0pt}{12pt plus 4pt minus 2pt}{6pt plus 2pt minus 2pt}

\usepackage{titling}
\setlength{\droptitle}{-.25cm}

%\setlength{\parindent}{0pt}
%\setlength{\parskip}{6pt plus 2pt minus 1pt}
%\setlength{\emergencystretch}{3em}  % prevent overfull lines 

\usepackage[T1]{fontenc}
\usepackage[utf8]{inputenc}

\usepackage{fancyhdr}
\pagestyle{fancy}
\usepackage{lastpage}
\renewcommand{\headrulewidth}{0.3pt}
\renewcommand{\footrulewidth}{0.0pt} 
\lhead{}
\chead{}
\rhead{\footnotesize Math 107: Elementary Statistics -- Fall 2016}
\lfoot{}
\cfoot{\small \thepage/\pageref*{LastPage}}
\rfoot{}

\fancypagestyle{firststyle}
{
\renewcommand{\headrulewidth}{0pt}%
   \fancyhf{}
   \fancyfoot[C]{\small \thepage/\pageref*{LastPage}}
}

%\def\labelitemi{--}
%\usepackage{enumitem}
%\setitemize[0]{leftmargin=25pt}
%\setenumerate[0]{leftmargin=25pt}




\makeatletter
\@ifpackageloaded{hyperref}{}{%
\ifxetex
  \usepackage[setpagesize=false, % page size defined by xetex
              unicode=false, % unicode breaks when used with xetex
              xetex]{hyperref}
\else
  \usepackage[unicode=true]{hyperref}
\fi
}
\@ifpackageloaded{color}{
    \PassOptionsToPackage{usenames,dvipsnames}{color}
}{%
    \usepackage[usenames,dvipsnames]{color}
}
\makeatother
\hypersetup{breaklinks=true,
            bookmarks=true,
            pdfauthor={ ()},
             pdfkeywords = {},  
            pdftitle={Math 107: Elementary Statistics},
            colorlinks=true,
            citecolor=blue,
            urlcolor=blue,
            linkcolor=magenta,
            pdfborder={0 0 0}}
\urlstyle{same}  % don't use monospace font for urls


\setcounter{secnumdepth}{0}

\usepackage{longtable}




\usepackage{setspace}

\title{Math 107: Elementary Statistics}
\author{Adam Loy}
\date{Fall 2016}


\begin{document}  

		\maketitle
		
	
		\thispagestyle{firststyle}

%	\thispagestyle{empty}


	\noindent \begin{tabular*}{\textwidth}{ @{\extracolsep{\fill}} lr @{\extracolsep{\fill}}}


E-mail: \texttt{\href{mailto:adam.m.loy@lawrence.edu}{\nolinkurl{adam.m.loy@lawrence.edu}}} & Web: \href{http://math107-lu.github.io}{\tt math107-lu.github.io}\\
Office Hours: M 3:10-5:00; TR 1:50-3:20; F 3:10-4:20  &  Class Hours: MWF 1:50-3:00\\
Office: 410 Briggs Hall  & Class Room: 420 Briggs Hall\\
	&  \\
	\hline
	\end{tabular*}
	
\vspace{2mm}
	


\section{What is this class about?}\label{what-is-this-class-about}

The world is experiencing a flood of data. Everywhere we look---from our
cell phones to our Amazon shopping carts---data are being collected.
Often, these data only give a partial picture of the phenomenon of
interest, so we must be able to learn from data in order to make
objective decisions in the presence of uncertainty. This course aims to
help you develop the tools to think with and about data in order to be
informed citizens in a data-centric world. More specifically, this
course will cover graphical and analytical tools to conduct data
analysis essential for gaining knowledge in almost any field.

\section{Expected Learning Outcomes}\label{expected-learning-outcomes}

After finishing this course, you should have:

\begin{itemize}
\tightlist
\item
  An understanding of the importance of data collection, the ability to
  recognize limitations in data collection methods, and an awareness of
  the role that data collection plays in determining the scope of
  inference.
\item
  The ability to use R to summarize data numerically and visually, and
  to perform straightforward data analysis procedures.
\item
  A solid conceptual understanding of key concepts such as the logic of
  statistical inference, estimation with intervals, and testing for
  significance.
\item
  The knowledge of which statistical methods to use in which situations,
  the technological expertise to use the appropriate method(s), and the
  understanding necessary to interpret the results correctly,
  effectively, and in context.
\item
  The ability to understand and think critically about data-based
  claims.
\item
  An awareness of the power of data.
\end{itemize}

\section{Course Logistics}\label{course-logistics}

\paragraph{Mathematical background:}\label{mathematical-background}

This is a statistics course. We will use mathematics as a tool, but will
concentrate on the statistical ideas, not on mathematics. To this end,
algebra II is the only mathematical prerequisite for this course, and I
will assume that everyone enrolled meets this requirement. If you need
help with these skills, I encourage you to utilize the Center for
Academic Success.

\paragraph{Required textbook:}\label{required-textbook}

\emph{Statistics: Unlocking the Power of Data}, Lock et al., 2013, John
Wiley \& Sons, ISBN 978-0-470-60187-7.

\paragraph{Computing:}\label{computing}

Modern statistical analysis is done in a computing environment, so this
course has a strong computational focus. We will use the R language,
which is free and open-source. Lawrence has an R Studio server that you
can access on campus by pointing a browser to
\url{https://rstudio.lawrence.edu/}. Alternatively, you can install R
and R Studio on your own computer.

\paragraph{Online discussion forum:}\label{online-discussion-forum}

We'll be using Piazza as our online forum. Piazza is your main venue to
ask questions, discuss problems, and help each other out. Piazza is a
question-and-answer system designed to streamline class discussion
outside of the classroom.

\section{Course Components}\label{course-components}

\paragraph{Preparation and study:}\label{preparation-and-study}

You must read the assigned sections of the text before we discuss them
in class so that you are already working with the ideas in advance of
hearing about them from me. In addition, review your lecture notes after
each lecture, carefully reconstructing for yourself the ideas,
arguments, and overall story that is developing. Coming to class for 70
minutes 3 times a week is not sufficient to learn statistics and
reorganize your thought processes.

\paragraph{Class attendance:}\label{class-attendance}

During class we will explore the statistical thought process through
lecture, discussion, and lab activities. Office hours are not
substitutes for class attendance.

\paragraph{Lab days:}\label{lab-days}

Weekly labs provide a chance for you to grapple with data and do
statistics. Labs will focus on computation and critical thinking about
correct and intelligent use of data and statistics. Additionally, labs
will help you learn R. Generally, lab assignments will be due at the
beginning of the next class period.

\paragraph{Homework:}\label{homework}

I will assign a few problems (\textasciitilde{}3-5) most Mondays and
Fridays. You should start working on the problems as soon as they are
assigned, and work on them a little (or a lot) every day. While the
homework will help you grapple with the material, you may need more
practice than the homework provides to master the material. The textbook
includes many problems at the end of each section. Working through these
problems (the odd problems have solutions in the back) will help you
solidify your understanding. I especially encourage you to work through
the ``Skill Builder'' problems.

\paragraph{Quizzes:}\label{quizzes}

There will be short quizzes on the reading/video assignments that are to
be completed on Moodle prior to 10:30 am on most class meeting days.
These are simple checks that you have completed the daily
reading/viewing and know the basic ideas. You should check Moodle
regularly so that you do not forget to complete them. There may be times
when quizzes are given at the beginning of class.

\paragraph{Exams:}\label{exams}

There will be two midterm exams and a final exam. The midterm exams are
(tentatively) scheduled for Wednesday, October 5, and Friday, October
28, during class. The final exam will be held on Tuesday November 22
from 11:30 a.m. to 2:00 p.m The date and time of the final exam is set
by the registrar, and under no circumstances will you be allowed to take
the final at a different time due to early travel plans.

\section{Course Policies}\label{course-policies}

\paragraph{Assessment Procedure:}\label{assessment-procedure}

Your final grade will be computed using the following weights. Your
overall score will be the maximum of the two computed scores, based on
the following two weighting schemes:

\begin{longtable}[]{@{}lll@{}}
\toprule
Component & Scheme 1 & Scheme 2\tabularnewline
\midrule
\endhead
Quizzes & 5\% & 5\%\tabularnewline
Labs & 10\% & 10\%\tabularnewline
Homework & 25\% & 25\%\tabularnewline
Exam 1 & 10\% & 20\%\tabularnewline
Exam 2 & 20\% & 20\%\tabularnewline
Final & 30\% & 20\%\tabularnewline
\bottomrule
\end{longtable}

Homework and classwork will be graded using the following 5-point scale:

\begin{longtable}[]{@{}ll@{}}
\toprule
\begin{minipage}[b]{0.08\columnwidth}\raggedright\strut
Points
\strut\end{minipage} &
\begin{minipage}[b]{0.86\columnwidth}\raggedright\strut
Characteristics
\strut\end{minipage}\tabularnewline
\midrule
\endhead
\begin{minipage}[t]{0.08\columnwidth}\raggedright\strut
5
\strut\end{minipage} &
\begin{minipage}[t]{0.86\columnwidth}\raggedright\strut
Almost all problems are essentially correct with no major conceptual
flaws. There may be some minor errors or calculation mistakes.
\strut\end{minipage}\tabularnewline
\begin{minipage}[t]{0.08\columnwidth}\raggedright\strut
4
\strut\end{minipage} &
\begin{minipage}[t]{0.86\columnwidth}\raggedright\strut
One problem is incomplete or contains a major conceptual flaw, but most
problems are essentially correct. There may also be some minor errors or
calculation mistakes.
\strut\end{minipage}\tabularnewline
\begin{minipage}[t]{0.08\columnwidth}\raggedright\strut
3
\strut\end{minipage} &
\begin{minipage}[t]{0.86\columnwidth}\raggedright\strut
At least two problems are incomplete or contain a major conceptual flaw,
but most problems are essentially correct. There may also be some minor
errors or calculation mistakes.
\strut\end{minipage}\tabularnewline
\begin{minipage}[t]{0.08\columnwidth}\raggedright\strut
2
\strut\end{minipage} &
\begin{minipage}[t]{0.86\columnwidth}\raggedright\strut
More than half the problems are incomplete or contain a major conceptual
flaw, but there is evidence that the student made a serious attempt to
solve most problems. Some parts of some problems are correct.
\strut\end{minipage}\tabularnewline
\begin{minipage}[t]{0.08\columnwidth}\raggedright\strut
1
\strut\end{minipage} &
\begin{minipage}[t]{0.86\columnwidth}\raggedright\strut
The assignment shows little progress toward a correct solution on any
problem, but there is evidence that some serious effort was put forth on
at least one problem.
\strut\end{minipage}\tabularnewline
\begin{minipage}[t]{0.08\columnwidth}\raggedright\strut
0
\strut\end{minipage} &
\begin{minipage}[t]{0.86\columnwidth}\raggedright\strut
The assignment is not turned in or contains no evidence of serious
effort on any problem.
\strut\end{minipage}\tabularnewline
\bottomrule
\end{longtable}

\paragraph{Homework deadlines:}\label{homework-deadlines}

The problems assigned on Monday are due Friday by 4:00 p.m., while those
assigned Friday are due Tuesday by 4:00 p.m. Problems are due in my
office and no late work will be accepted. I understand that this policy
is strict, so I will drop your two lowest scores when computing your
homework average.

\paragraph{Classroom Culture:}\label{classroom-culture}

If you would rather be talking, sleeping, reading the news, listening to
music, or texting, I suggest that you do that somewhere much more
comfortable than the classroom. When you attend class, please arrive on
time and stay engaged throughout the entire class.

\paragraph{Honor Code:}\label{honor-code}

\begin{quote}
\emph{No Lawrence student will unfairly advance their own academic
performance or in any way limit or impede the academic pursuits of other
students of the Lawrence community.}
\end{quote}

All students are expected to uphold Lawrence University's Honor Code.
All work on quizzes and exams must be your own. You may collaborate on
homework, but you must submit your own assignment that reflects your own
thinking, work and organization. Any assignment you submit for a grade
should be your own work, and not a facsimile of a classmate's work,
which would constitute academic dishonesty. To check if your homework
meets this standard, imagine I asked you to explain your reasoning for
each problem---you should be able to do so with ease using language
similar to your submission. All written work must be accompanied by a
reaffirmation of the Honor Code. Finally, cell phones will be prohibited
during exams.

\paragraph{Disability Policy:}\label{disability-policy}

Lawrence University is committed to providing reasonable accommodations
for students with disabilities. Students establish eligibility and
request accommodations through the Center for Academic Success. View the
Accessibility Services web page at go.lawrence.edu/cas for more
information.

\paragraph{Healthy Balance:}\label{healthy-balance}

All members of the Lawrence community---students, staff, and
faculty---have the responsibility to promote balance in their lives by
making thoughtful choices. Balance results from two skills: avoiding
imbalance through careful planning, and managing and containing
imbalance when it occurs. This course will be demanding, but should not
overwhelm your academic (let alone whole) life. If it threatens to, come
talk to me, a tutor, friend, counselor, or advisor.




\end{document}

\makeatletter
\def\@maketitle{%
  \newpage
%  \null
%  \vskip 2em%
%  \begin{center}%
  \let \footnote \thanks
    {\fontsize{18}{20}\selectfont\raggedright  \setlength{\parindent}{0pt} \@title \par}%
}
%\fi
\makeatother
